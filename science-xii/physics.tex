\documentclass[a4paper,10pt]{report}
\usepackage[utf8]{inputenc}
\usepackage{textcomp}

% Title Page
\title{HSEB Old is Gold}
\author{HSEB NOTES}

\begin{document}
\maketitle

\begin{abstract}
\end{abstract}

\chapter{Wave and Optics}

\section{Wave Motion}
 \subsection{Short Questions}
  \begin{enumerate}
   \item Differentiate between forced vibration and free vibration. \textbf{[1.b., 2052]}
   \item How are stationary waves formed? \textbf{[1.b., 2055]}
   \item If you are walking on the moon surface, can you hear the cracking sound behind you? Explain. 
    \textbf{[1.c., 2058]}
   \item Why are longitudinal waves not polarised? \textbf{[2.b., 2066 Supplementary]}
   \item Do sound waves undergo reflection, refraction, and polarisation phenomena? Explain. 
    \textbf{[2.c., 2062]}
   \item Which types of wave propagate in liquid, explain. \textbf{[1.c., 2063]}
   \item Longitudinal waves cannot be polarised. Why? \textbf{[3.b., 2067 1$^{st}$ exam]}
   \item Distinguish between light waves and sound waves. \textbf{[3.a., 2068 1$^{st}$ exam]}
   \item A radio station broadcasts at 800 KHz. What will be the wavelength of the wave? 
    \textbf{[3.b., set 'B' 2069]}
  \end{enumerate}

 \subsection{Long Questions}
  \begin{enumerate}
   \item What are stationary waves? Derive an equation for such wave. \textbf{[3.a.(or), 2060 Supplementary]}
   \item Use the principle of superposition of two waves to find the position of displacement nodes 
    and anti-nodes in a standing wave. \textbf{[5.a., 2061]}
   \item State and explain the stationary wave. \textbf{[6.a., 2063]}
   \item What is meant by interference of sound? How the constructive and destructive interference are formed? 
    \textbf{[4.a.(or), 2063 Supplementary]}
   \item What is the principle of superposition as applied to wave motion? 
    Discuss the result of superposing two waves of equal amplitude and same frequency travelling in opposite 
    direction. \textbf{[7.b., 2068 2$^{nd}$ exam]}
  \end{enumerate}

 \subsection{Numerical Problems}
  \begin{enumerate} 
   \item A wave has the equation: $Y=0.02 sin(30t-4x)$
    \begin{enumerate} 
     \item Its frequency, speed and wavelength.
     \item The equation of wave with double amplitude but travelling in the opposite direction.
    \end{enumerate} \textbf{[Ans. 4.77Hz, 1.57m, 7.5ms$^{-1}$][3(or), 2053]}
  \end{enumerate}

\section{Mechanical Waves}
 \subsection{Short Questions}
  \begin{enumerate}
   \item Why sound made at a distance can be heard distinctly at night than in the day time? 
    \textbf{[1.c., 2052; $\sim$2.c., 2060; $\sim$1.c., 2065; $\sim$1.c., 2066]}
   \item Is velocity of sound more in damp air or in dry air? Explain. \textbf{[1.b., 2054; $\sim$1.c., 2056]}
   \item Do sound waves need mediums to travel from one point to another point in space? What properties of the
    medium are relevant? \textbf{[1.c., 2062]}
   \item Although the density of solid is high, the velocity of sound is great in solid, explain. 
    \textbf{[2.a., 2063; $\sim$2.d., 2064]}
  \end{enumerate}

 \subsection{Long Questions}
  \begin{enumerate}
   \item Discuss the effect of pressure, temperature and density of elastic medium on the velocity of sound. 
    \textbf{[2., 2052; ]}
   \item What is Newton's formula for the velocity of sound? What correction was made by Laplace? 
    \textbf{[2., 2053; $\sim$2., 2055; $\sim$2., 2056; $\sim$5.a.(or), 2061; $\sim$3.a.(or), 2062 Supplementary;
     $\sim$5.a.(or), 2065; $\sim$7.a., 2068 1$^{st}$ exam; $\sim$7.a., 2068 1$^{st}$; $\sim$7.b., Set 'B', 2069]}
   \item Derive an expression for the velocity of sound in a medium by dimensional method. Discuss the
    effect of change in pressure and temperature on the velocity of sound in air. \textbf{[5.a., 2060]}
   \item Write down the factors on which the velocity of sound in air depends with necessary explanation. 
    \textbf{[5.a., 2066; 7.b., set B, 2069]}
  \end{enumerate}
  
 \subsection{Numerical Problems}
  \begin{enumerate}
   \item The interval between the flash of lightning and the sound of thunder is 2 seconds, when temperature 
    is 10\textcelsius{}. How far is the storm if the velocity of sound in air at 0\textcelsius{} is 330 ms$^{-1}$?
    \textbf{[Ans: 336 m][3.(or), 2052]}
   \item A man standing at one end of a closed corridor 57 m long blows a short blast on a whistle. 
    He found that the time from the blast to the sixth echo was 2 seconds. If the temperature was 17\textcelsius{}, 
    what was the velocity of sound at 0\textcelsius{}? \textbf{[Ans: 331.8 ms$^{-1}$][5.b., 2057]}
   \item At what temperature, the velocity of sound in air is increased by 50\%{} to that at 27\textcelsius{}?
    \textbf{[Ans: 675 K][5.b., 2058]}
   \item A source of sound of frequency 512 Hz emits waves of wavelength 670 mm in air at 20\textcelsius{}. What is 
    the velocity of sound in air at this temperature? What would be the wavelength of sound from the source in air 
    at 0\textcelsius{}? \textbf{[Ans: 343 ms$^{-1}, 644\times10^{-3}$m]}
   \item When a detonator is exploded on a railway line, an observer standing on the rail 2 KM away hears two sounds. 
    What is the time interval between them? (Young's modulus of steel = $2\times10^{11} Nm^{-2}$, density of steel =
    $8\times10^{3} KG m^{-3}$ density of air = $1.4 KG m^{-3}$, $\gamma$ - for air = 1.4, atmospheric pressure = 
    $10^5 Nm^{-2}$) \textbf{[Ans: 5.92 seconds][11., set A, 2069]}
  \end{enumerate}

\section{Wave in Pipes and Strings}
 \subsection{Short Questions}
  \begin{enumerate}
   \item The frequency of fundamental note of an open organ pipe is double than for closed pipe of same length. Why?
    \textbf{[1.a., 2052]}
   \item The frequency of organ pipe changes with temperature. Does it increase with increase in temperature? 
    \textbf{[1.a., 2053; $\sim$2.c., 2061]}
   \item Explain why soldiers are ordered to break steps while crossing a bridge. \textbf{[1.c., 2053]}
   \item Why is an end correction necessary for an organ pipe? \textbf{1.a., 2054]}
   \item Relate the fundamental note with overtones for an open pipe. \textbf{[1.a., 2055]}
   \item Why is sonometer box hollow from the inside? \textbf{[1.a., 2056]}
   \item What do you mean by resonance? \textbf{[1.d., 2057]}
   \item One of the 'Nine Jewels' of Emperor Akbar, widely known as Tansen, the king of Music was able to break a glass
    by singing the appropriate note. What physical phenomenon could account for this? \textbf{[2.d., 2059]}
   \item Why is loud sound heard at resonance? \textbf{[1.a., 2060 Supplementary]}
   \item Explain why sound produced by a vibrating fork becomes louder when its stem is placed in contact with a table.
    \textbf{[2.a., 2060 Supplementary]}
   \item Why are rubbers used to absorb vibration? \textbf{[1.d., 2063]}
   \item When the tension in a given stretched string is increased by four times, by what factor does the velocity of
    transverse wave in the string change? \textbf{[3.a., 2067 1$^{st}$ exam]}
   \item Would you expect the pitch of an organ pipe to change with an increase in temperature? How? 
    \textbf{[3.a., 2067 2$^{nd}$ exam]}
   \item Is the wave speed the same as the speed of any part of the string for transverse waves? Explain the difference
    between these two speeds. \textbf{[3.b., 2067 2$^{nd}$ exam]}
   \item If the frequency of a fundamental note of a closed pipe and that of an open pipe are the same, what will be the
    ratio between their lengths? \textbf{[3.b., 2068 1$^{st}$ exam]}
   \item Is it possible to have a longitudinal wave on a stretched string? Why or why not? \textbf{[3.b., 2068 2$^{nd}$ exam]}
   \item What happens to the frequency of transverse vibration of a stretched string if its tension is halved and the 
    area of cross section of the string is doubled? \textbf{[3.b., set A, 2069]}
  \end{enumerate}
 
 \subsection{Long Questions}
  \begin{enumerate}
   \item Describe the various modes of vibrations of the air column in an organ pipe. \textbf{[2., 2054]}
   \item Describe the various modes of vibrations of the air column in a closed organ pipe. \textbf{[5.a.(or), 2057;
    6.a.(or), 2063]}
   \item Prove that the both types of harmonics, odd and even, can be produced in an organ pipe open at both ends. 
    \textbf{[5.a.(or), 2058; $\sim$7.a., set A, 2069]}
   \item State the laws of transverse vibrations of string. Using only dimensions, show that the speed of propagation of a
    transverse wave depends only on tension and mass per unit length. \textbf{[5.a.(or), 2059; $\sim$6.a.(or), 2062]}
   \item Describe the natural modes of vibration of air in an organ pipe closed at one end and explain the term ``end 
    correction``. \textbf{[3.a., 2060 Supplementary]}
   \item What do you mean by stationary wave? Discuss the possible mode of vibration of column of air in an open pipe. 
    \textbf{[3.a., 2062 Supplementary]}
   \item Describe the resonance tube experiment to determine the velocity of sound in air at N.T.P.? 
    \textbf{[4.a., 2063 Supplementary]}
   \item State and explain principle of superposition and formation of stationary waves. Show that the frequency of the 
    fundamental note of a closed organ pipe is half as compared to that of an open pipe of the same length. \textbf{[
    5.a.(or), 2064]}
   \item What do you understand by ''harmonics`` and ''overtone`` in the case of organ pipe. Also prove that only odd
    harmonics are used produced in closed ended organ pipe. \textbf{[5.a., 2065]}
   \item Describe an experiment giving the necessary theory by which the speed of sound in air may be determined using
    resonance air column method. \textbf{[7.a., 2067 2$^{nd}$ exam]}
   \item What is meant by resonance? Explain in detail how you would use sonometer to determine frequency of a given tuning
    fork. \textbf{[7.a., 2068 2$^{nd}$ exam]}
   \item Show that both harmonics, odd and even, can be produced in an open organ pipe. What is end correction? \textbf{[
    7.a., set A, 2069]}
  \end{enumerate}
 
 \subsection{Numerical Problems}
  \begin{enumerate}
   \item A steel wire of 2m length, 3g mass is under tension of 500N and is tied down ad both ends. Calculate the frequency
    and wavelength for fundamental mode of vibration. \textbf{[Ans: 144 Hz, 4m][3., 2052]}
   \item A wire of diameter 0.04 cm and made of steel of density 8000 kgm$^{-3}$ is under constant tension of 80N. What
    length of this wire should be plucked to cause it to vibrate with a frequency of 840 Hz? \textbf{[Ans: 0.168 m][3., 2053;
    5.b., 2060]}
   \item A piano string has length of 2m and a density of 8000 kgm$^{-3}$. When the tension in the string produces a string
    of 1\%{}, the fundamental note obtained from the string in the transverse vibrations in 170Hz. calculate Young's modulus
    for the material of the string. \textbf{[Ans: $3.7\times10^{10} Nm^{-2}$][3., 2055; 5.b., 2061]}
   \item An organ pipe is tuned to a frequency of 440 Hz when the temperature is 27\textcelsius{}. Find its frequency when
    the temperature drops to 0\textcelsius{}. Assume both ends of the pipe open. \textbf{[Ans: 400.4Hz][3.(or), 2056]}
   \item Resonance was observed with an air column when its length was 16 cm and again when it was 50 cm. If the frequency of
    the tuning fork is 512 Hz, calculate:
    \begin{enumerate}
     \item the velocity of sound
     \item end correction
    \end{enumerate}
    \textbf{[Ans: a. 348.16 ms$^{-1}$ b. 0.01 m][3.b., 2062 Supplementary]}
   \item A uniform tube 60 cm long stands vertically with its lower end dipping into water. When the length above the water 
    is 14.8 cm and again when it is 48 cm, the tube resounds to a vibrating tuning fork of frequency 512 Hz. Find the lowest 
    frequency to which the tube will resound when it is open at both ends. \textbf{[Ans: -267 Hz][5.b., 2066]}
   \item An open pipe 30 cm long and closed pipe 23 cm long, both of the same diameter, are each sounding its first overtone,
    and are in resonance. What is the end correction for these pipes? \textbf{[Ans: 1 cm][11., 2067 1$^{st}$ exam; 11., 2068
    2${nd}$ exam; 11., set B, 2069]}
   \item AN observer travelling with constant velocity of 20 m/s pass close to a stationary source of sound and notices that
    there is a change of frequency of 50 Hz as he passed the source. What is the frequency of the source? Speed of the sound 
    in air = 340 m/s. \textbf{[Ans: 425 Hz][11., 2067 2$^{nd}$ exam]}
   \item A piano string 1.5 m long is made of steel of density 7800 $kg/m^{3}$ and Young's modulus $2\times10^{11} Nm^{-2}$. 
    It is maintained at a tension which produced an elastic strain of 1\%{} in the string. Calculate the frequency of transverse
    vibration of the string when it is vibrating in second mode of vibration. \textbf{[Ans: 168.8Hz][11., 2068 1$^{st}$ exam]}
  \end{enumerate}

\section{Acoustic Phenomena}
 \subsection{Short Questions}
  \begin{enumerate}
   \item How are beats produced? What is beat frequency? \textbf{[1.b., 2053; $\sim$1.d., 2061]}
   \item What are ultrasonic and infrasonic? \textbf{[1.c., 2054]}
   \item distinctly between ultrasonic and supersonic. \textbf{[1.c., 2055]}
   \item How is that one can recognise a friend from his voice without seeing him? \textbf{[1.b., 2056]}
   \item Why is the roaring of a line different than the humming of a mosquito? \textbf{[2.d., 2057]}
   \item Why does an empty vessel produce more sound than a filled one? \textbf{[2.d., 2058; 3.a., set A, 2069]}
   \item What do you mean by the term threshold of hearing? \textbf{[1.c., 2059]}
   \item Whistle of an approaching train is shriller. Why? \textbf{[1.c., 2060]}
   \item Why we cannot hear echo in small room? \textbf{[1.a., 2062 Supplementary}
   \item What is the good and bad effect of echo? \textbf{[1.a., 2063 Supplementary]}
   \item A violin note and a guitar note may have the same frequency yet we can distinguish them just by hearing them, why?
    \textbf{[2.a., 2063 Supplementary]}
   \item Bells are made of metal and not of wood, why? \textbf{[1.c., 2064]}
   \item Explain with a figure the meaning of beats. \textbf{[2.d., 2065]}
   \item Why are all string instruments provided with hollow boxes? \textbf{[2.c., 2066]}
   \item Which has more direct influence on the loudness of a sound wave: the displacement amplitude or the pressure amplitude?
    Explain your reasoning. \textbf{[3.a., 2068 2$^{nd}$ exam]}
   \item is there a physical difference between intensity and intensity level of wave? How are these quantities related?
    \textbf{[3.a., set B, 2069]}
  \end{enumerate}
  
 \subsection{Long Questions}
  \begin{enumerate}
   \item Deduce the expressions for the frequency heard by an observer, when the observer is approaching the stationary sound
    source. \textbf{[2.(or), 2055]}
   \item What is Doppler's effect? Derive an expression for the apparent frequency received by a stationary observer when a 
    source is moving away from him. \textbf{[5.a., 2057]}
   \item What do you mean by intensity and intensity level of sound? Define bel and decibel. \textbf{[5.a., 2058]}
   \item What is Doppler's effect? Derive the change in frequency when an observer moves towards a stationary source. \textbf{
    [5.a., 2059]}
   \item What are beats? Prove that the number of beats per second is equal to the difference between the frequencies of two 
    superposing waves. \textbf{[5.a.(or), 2060]}
   \item Discuss the phenomenon of Doppler's effect. Find the change in frequency when a moving source of sound passes a stationary
    observer. \textbf{[6.a., 2062; $\sim$7.a., set B, 2069]}
   \item Define intensity of sound. Show that the intensity of sound for a given frequency is directly proportional to the square
    of amplitude of vibration. \textbf{[5.a., 2064]}
   \item Define the intensity of sound and prove that $I = \frac 12 \rho vr^2 \omega^2$ where the symbols have their usual meaning.
    \textbf{[5.a.(or), 2066]}
   \item What are beats? Obtain the expression for the beat frequency when beats are produced by superposing two waves of slightly
    different frequencies. \textbf{[7.a., 2067 1$^{st}$ exam]}
   \item Define intensity and deduce it in terms of amplitude of vibration, density of medium, angular velocity of the wave. 
    \textbf{[3.b., 2067 2$^{nd}$ exam]}
   \item What is Doppler's effect? Deduce an expression for the apparent frequency heard by a stationary observer when a source
    approaches towards him. \textbf{[7.b., 2068 1$^{st}$ exam]}
   \item What is Doppler's effect? Obtain an expression for the apparent pitch when an observer moves towards a stationary source.
    \textbf{[7.b., 2068 1$^{st}$ exam]}
  \end{enumerate}


\end{document}          
